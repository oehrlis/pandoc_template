%%
% -----------------------------------------------------------------------------
% Trivadis AG, Infrastructure Managed Services
% Saegereistrasse 29, 8152 Glattbrugg, Switzerland
% -----------------------------------------------------------------------------
% Name.......: trivadis.tex
% Author.....: Stefan Oehrli (oes) stefan.oehrli@trivadis.com
% Editor.....: Stefan Oehrli
% Date.......: 2018.11.01
% Revision...: --
% Purpose....: LaTeX template for pandoc
% Notes......: For usage information and examples visit the GitHub page of this 
%              template: https://github.com/oehrlis/pandoc_template
% Reference..: https://github.com/oehrlis/pandoc_template
% License....: GPL-3.0+
% -----------------------------------------------------------------------------
% Modified :
% see git revision history with git log for more information on changes
% -----------------------------------------------------------------------------
%%

\PassOptionsToPackage{unicode=true}{hyperref} % options for packages loaded elsewhere
\PassOptionsToPackage{hyphens}{url}
\PassOptionsToPackage{dvipsnames,svgnames*,table}{xcolor}

\documentclass[a4paper,,tablecaptionabove]{scrartcl}



  \usepackage{lmodern}


\usepackage{amssymb,amsmath}
\usepackage{ifxetex,ifluatex}
\usepackage{fixltx2e} % provides \textsubscript
\ifnum 0\ifxetex 1\fi\ifluatex 1\fi=0 % if pdftex
  \usepackage[T1]{fontenc}
  \usepackage[utf8]{inputenc}
  \usepackage{textcomp} % provides euro and other symbols
\else % if luatex or xelatex

  \usepackage{unicode-math}

\defaultfontfeatures{Ligatures=TeX,Scale=MatchLowercase}







\fi


% use upquote if available, for straight quotes in verbatim environments
\IfFileExists{upquote.sty}{\usepackage{upquote}}{}

% use microtype if available
\IfFileExists{microtype.sty}{%
  \usepackage[]{microtype}
  \UseMicrotypeSet[protrusion]{basicmath} % disable protrusion for tt fonts
}{}

  \IfFileExists{parskip.sty}{%
    \usepackage{parskip}
  }{% else
    \setlength{\parindent}{0pt}
    \setlength{\parskip}{6pt plus 2pt minus 1pt}
  }



\usepackage{hyperref}
\hypersetup{
      pdftitle={EUS, Kerberos, SSL and OUD a guideline},
        pdfauthor={Stefan Oehrli},
            pdfborder={0 0 0},
    breaklinks=true}

\urlstyle{same}  % don't use monospace font for urls

  \usepackage[margin=2.5cm,includehead=true,includefoot=true,centering]{geometry}

% load graphicx any way
  \usepackage[export]{adjustbox}
  \usepackage{graphicx}


  \usepackage{listings}
  \newcommand{\passthrough}[1]{#1}



  \usepackage{longtable,tabularx,booktabs}
      % Fix footnotes in tables (requires footnote package)
    \IfFileExists{footnote.sty}{\usepackage{footnote}\makesavenoteenv{longtable}}{}
  



\setlength{\emergencystretch}{3em}  % prevent overfull lines
\providecommand{\tightlist}{\setlength{\itemsep}{0pt}\setlength{\parskip}{0pt}}

  \setcounter{secnumdepth}{0}

      % Redefines (sub)paragraphs to behave more like sections
    \ifx\paragraph\undefined\else
    \let\oldparagraph\paragraph
    \renewcommand{\paragraph}[1]{\oldparagraph{#1}\mbox{}}
    \fi
    \ifx\subparagraph\undefined\else
    \let\oldsubparagraph\subparagraph
    \renewcommand{\subparagraph}[1]{\oldsubparagraph{#1}\mbox{}}
    \fi
  

% Make use of float-package and set default placement for figures to H
\usepackage{float}
\floatplacement{figure}{H}






  \title{EUS, Kerberos, SSL and OUD a guideline}

  \providecommand{\subtitle}[1]{}
  \subtitle{Demo Scripts, Examples and Exercises}

  \author{Stefan Oehrli}

\date{2018 November 01}


%% - Begin added ---------------------------------------------------------
% No language specified? take American English.
  \ifnum 0\ifxetex 1\fi\ifluatex 1\fi=0 % if pdftex
    \usepackage[shorthands=off,main=english]{babel}
      \else
          % See issue https://github.com/reutenauer/polyglossia/issues/127
      \renewcommand*\familydefault{\sfdefault}
        % load polyglossia as late as possible as it *could* call bidi if RTL lang (e.g. Hebrew or Arabic)
    \usepackage{polyglossia}
    \setmainlanguage[]{english}
      \fi

%
% colors
\usepackage[]{xcolor}

%
% Trivadis colors
\definecolor{tvdgray}{gray}{0.9}
\definecolor{tvdgray2}{gray}{0.8}
\definecolor{tvdred}{RGB}{204,0,0}
\definecolor{tvdyellow}{RGB}{255,128,0}
\definecolor{tvdgreen}{RGB}{51,192,0}

% listing colors
\definecolor{listing-background}{HTML}{F7F7F7}
\definecolor{listing-rule}{HTML}{B3B2B3}
\definecolor{listing-numbers}{HTML}{B3B2B3}
\definecolor{listing-text-color}{HTML}{000000}
\definecolor{listing-keyword}{HTML}{435489}
\definecolor{listing-identifier}{HTML}{435489}
\definecolor{listing-string}{HTML}{00999A}
\definecolor{listing-comment}{HTML}{8E8E8E}
\definecolor{listing-javadoc-comment}{HTML}{006CA9}

% for the background color of the title page  
  % make use of textpos package
  \usepackage[absolute]{textpos}
  \usepackage{pagecolor}
  \usepackage{afterpage}

% TOC depth and 
% section numbering depth
\setcounter{tocdepth}{3}

% line spacing
  \usepackage{setspace}
  \setstretch{1.2}

% break urls
\PassOptionsToPackage{hyphens}{url}

% When using babel or polyglossia with biblatex, loading csquotes is recommended 
% to ensure that quoted texts are typeset according to the rules of your main language.
%
\usepackage{csquotes}

% captions
\definecolor{caption-color}{HTML}{777777}
\usepackage[font={stretch=1.2}, textfont={color=caption-color}, position=top, skip=4mm, labelfont=bf, singlelinecheck=false, justification=raggedright]{caption}
\setcapindent{0em}
\captionsetup[longtable]{position=above}

% blockquote
\definecolor{blockquote-border}{RGB}{221,221,221}
\definecolor{blockquote-text}{RGB}{119,119,119}
\usepackage{mdframed}
\newmdenv[rightline=false,bottomline=false,topline=false,linewidth=3pt,linecolor=blockquote-border,skipabove=\parskip]{customblockquote}
\renewenvironment{quote}{\begin{customblockquote}\list{}{\rightmargin=0em\leftmargin=0em}%
\item\relax\color{blockquote-text}\ignorespaces}{\unskip\unskip\endlist\end{customblockquote}}

% Source Sans Pro as the de­fault font fam­ily
% Source Code Pro for monospace text
%
% 'default' option sets the default 
% font family to Source Sans Pro, not \sfdefault.
%
  %\usepackage[default]{sourcesanspro}
  \usepackage[scaled]{uarial}
  \renewcommand\familydefault{\sfdefault} 
  \usepackage{sourcecodepro}

% heading color
\definecolor{heading-color}{RGB}{40,40,40}
\addtokomafont{section}{\color{heading-color}}
% When using the classes report, scrreprt, book, 
% scrbook or memoir, uncomment the following line.
%\addtokomafont{chapter}{\color{heading-color}}

% variables for title and author
\usepackage{titling}
\title{EUS, Kerberos, SSL and OUD a guideline}
\author{Stefan Oehrli}

% tables
  \definecolor{table-row-color}{HTML}{F5F5F5}
  \definecolor{table-rule-color}{HTML}{999999}

  %\arrayrulecolor{black!40}
  \arrayrulecolor{table-rule-color}     % color of \toprule, \midrule, \bottomrule
  %\arrayrulecolor{tvdgray2}     % color of \toprule, \midrule, \bottomrule
  \setlength\heavyrulewidth{0.3ex}      % thickness of \toprule, \bottomrule
  \renewcommand{\arraystretch}{1.3}     % spacing (padding)

  % Reset rownum counter so that each table
  % starts with the same row colors.
  % https://tex.stackexchange.com/questions/170637/restarting-rowcolors
  \let\oldlongtable\longtable
  \let\endoldlongtable\endlongtable
  \renewenvironment{longtable}{
  \rowcolors{3}{}{table-row-color!100}  % row color
  \oldlongtable} {
  \endoldlongtable
  \global\rownum=0\relax}

  % Unfortunately the colored cells extend beyond the edge of the 
  % table because pandoc uses @-expressions (@{}) like so: 
  %
  % \begin{longtable}[]{@{}ll@{}}
  % \end{longtable}
  %
  % https://en.wikibooks.org/wiki/LaTeX/Tables#.40-expressions

% remove paragraph indention
\setlength{\parindent}{0pt}
\setlength{\parskip}{6pt plus 2pt minus 1pt}
\setlength{\emergencystretch}{3em}  % prevent overfull lines

% Listings
  \lstdefinestyle{eisvogel_listing_style}{
    language         = java,
          xleftmargin      = 0.6em,
      framexleftmargin = 0.4em,
        backgroundcolor  = \color{listing-background},
    basicstyle       = \color{listing-text-color}\small\ttfamily{}\linespread{1.15}, % print whole listing small
    breaklines       = true,
    frame            = single,
    framesep         = 0.6mm,
    rulecolor        = \color{listing-rule},
    frameround       = ffff,
    tabsize          = 4,
    numberstyle      = \color{listing-numbers},
    aboveskip        = 1.0em,
    belowcaptionskip = 1.0em,
    keywordstyle     = \color{listing-keyword}\bfseries,
    classoffset      = 0,
    sensitive        = true,
    identifierstyle  = \color{listing-identifier},
    commentstyle     = \color{listing-comment},
    morecomment      = [s][\color{listing-javadoc-comment}]{/**}{*/},
    stringstyle      = \color{listing-string},
    showstringspaces = false,
    escapeinside     = {/*@}{@*/}, % Allow LaTeX inside these special comments
    literate         =
    {á}{{\'a}}1 {é}{{\'e}}1 {í}{{\'i}}1 {ó}{{\'o}}1 {ú}{{\'u}}1
    {Á}{{\'A}}1 {É}{{\'E}}1 {Í}{{\'I}}1 {Ó}{{\'O}}1 {Ú}{{\'U}}1
    {à}{{\`a}}1 {è}{{\'e}}1 {ì}{{\`i}}1 {ò}{{\`o}}1 {ù}{{\`u}}1
    {À}{{\`A}}1 {È}{{\'E}}1 {Ì}{{\`I}}1 {Ò}{{\`O}}1 {Ù}{{\`U}}1
    {ä}{{\"a}}1 {ë}{{\"e}}1 {ï}{{\"i}}1 {ö}{{\"o}}1 {ü}{{\"u}}1
    {Ä}{{\"A}}1 {Ë}{{\"E}}1 {Ï}{{\"I}}1 {Ö}{{\"O}}1 {Ü}{{\"U}}1
    {â}{{\^a}}1 {ê}{{\^e}}1 {î}{{\^i}}1 {ô}{{\^o}}1 {û}{{\^u}}1
    {Â}{{\^A}}1 {Ê}{{\^E}}1 {Î}{{\^I}}1 {Ô}{{\^O}}1 {Û}{{\^U}}1
    {œ}{{\oe}}1 {Œ}{{\OE}}1 {æ}{{\ae}}1 {Æ}{{\AE}}1 {ß}{{\ss}}1
    {ç}{{\c c}}1 {Ç}{{\c C}}1 {ø}{{\o}}1 {å}{{\r a}}1 {Å}{{\r A}}1
    {€}{{\EUR}}1 {£}{{\pounds}}1 {«}{{\guillemotleft}}1
    {»}{{\guillemotright}}1 {ñ}{{\~n}}1 {Ñ}{{\~N}}1 {¿}{{?`}}1
    {…}{{\ldots}}1 {≥}{{>=}}1 {≤}{{<=}}1 {„}{{\glqq}}1 {“}{{\grqq}}1
    {”}{{''}}1
  }
  \lstset{style=eisvogel_listing_style}

  \lstdefinelanguage{XML}{
    morestring      = [b]",
    moredelim       = [s][\bfseries\color{listing-keyword}]{<}{\ },
    moredelim       = [s][\bfseries\color{listing-keyword}]{</}{>},
    moredelim       = [l][\bfseries\color{listing-keyword}]{/>},
    moredelim       = [l][\bfseries\color{listing-keyword}]{>},
    morecomment     = [s]{<?}{?>},
    morecomment     = [s]{<!--}{-->},
    commentstyle    = \color{listing-comment},
    stringstyle     = \color{listing-string},
    identifierstyle = \color{listing-identifier}
  }

% header and footer
  \usepackage{fancyhdr}
  \usepackage{lastpage}
  \pagestyle{fancy}
  \fancyhead{}
  \fancyfoot{}

  \newcommand*{\TVDLogo}{\includegraphics[width=0.2\textwidth]{/root/.pandoc/images/TVDLogo2019.eps}}

  \lhead[ \TVDLogo ]{}
  \chead[]{}
  \rhead[]{ \TVDLogo }

  \lfoot[\thepage\ / \pageref*{LastPage}]{EUS, Kerberos, SSL and OUD a guideline}
  \cfoot[]{}
  \rfoot[EUS, Kerberos, SSL and OUD a guideline]{\thepage\ / \pageref*{LastPage}}

  \renewcommand{\headrulewidth}{0.4pt}
  \renewcommand{\footrulewidth}{0.4pt}

%% - End added -----------------------------------------------------------

%% - Begin Document ------------------------------------------------------
\begin{document}
%% - Begin titlepage -----------------------------------------------------
  %\thispagestyle{fancyplain2}
  \begin{titlepage}
  
  % TVD Boxes
  \begin{flushleft}
  %  \color{tvdred}\rule{2,5cm}{2,5cm} \hspace{2mm} \color{tvdgray}\rule{2,5cm}{2,5cm} \hspace{2mm} \rule{2,5cm}{2,5cm} 
  \end{flushleft}

      \begin{flushright}
      \includegraphics[width=5.12cm, right]{/root/.pandoc/images/TVDLogo2019.eps}
    \end{flushright}
  
  \begin{flushright}
    \vfill
    \noindent {\huge \textbf{\textsf{EUS, Kerberos, SSL and OUD a guideline}}}\\
        \bigskip
    {\Large \textsf{Demo Scripts, Examples and Exercises}}\\
    
    \bigskip

    \begin{flushright}
      \textbf{ 
        2018 November 01,
        Version 0.9 }\\
    \end{flushright}
    \vfill
  \end{flushright}

  %% TVD Location
  \begin{flushright}
     \textit{Trivadis AG\\} 
     \textit{Sägereistrasse 29\\} 
     \textit{8152} 
     \textit{Glattbrugg} 
    \par
  \end{flushright}

  %% TVD Contact
  \begin{flushright}
    
    
     \textit{info@trivadis.com\\} 
     \textit{+41 58 459 55 55\\} 
  \end{flushright}
  \end{titlepage}
%% - End Titlepage -------------------------------------------------------

    

  
      {
            \setcounter{tocdepth}{3}
      \tableofcontents
              \newpage
          }
  


\hypertarget{demos-eus-kerberos-ssl-and-oud-a-guideline}{%
\section{Demos EUS, Kerberos, SSL and OUD a
guideline}\label{demos-eus-kerberos-ssl-and-oud-a-guideline}}

A couple of demo's for the TechEvent presentation \emph{EUS, Kerberos,
SSL and OUD a guideline}. Be aware, that the code can not be used
copy/past in all environments due to limitations on the line breaks.

Demos are shown on an Oracle 18c Docker based database.

\begin{lstlisting}[language=bash]
docker run --detach --name te2018_eusdb \
  --volume /data/docker/volumes/te2018_eusdb:/u01 \
  -e ORACLE_SID=TE18EUS \
  -p 1521:1521 -p 5500:5500 \
  --hostname te2018_eusdb.postgasse.org \
  --dns 192.168.56.70 \
  --dns-search postgasse.org \
  oracle/database:18.3.0.0
\end{lstlisting}

Create user and roles

\begin{lstlisting}[language=SQL]
CREATE ROLE tvd_connect;
GRANT CREATE SESSION TO tvd_connect;
GRANT select ON v_$session TO tvd_connect;
CREATE USER SOE_KERBEROS IDENTIFIED EXTERNALLY AS 'soe@POSTGASSE.ORG';
GRANT tvd_connect TO SOE_KERBEROS;
\end{lstlisting}

\begin{longtable}[]{@{}lll@{}}
\toprule
ID & Test & Comment\tabularnewline
\midrule
\endhead
1 & wieso & halt text\tabularnewline
2 & wieso & halt text\tabularnewline
3 & wieso & halt text\tabularnewline
4 & wieso & halt text\tabularnewline
5 & wieso & halt text\tabularnewline
\bottomrule
\end{longtable}

\hypertarget{password-verifier}{%
\subsection{Password Verifier}\label{password-verifier}}

Clean up and remove the old users.

\begin{lstlisting}[language=SQL]
DROP USER user_10g;
DROP USER user_11g;
DROP USER user_12c;
DROP USER user_all;
\end{lstlisting}

Create 4 dedicated test user and grant them \emph{CREATE SESSION}.

\begin{lstlisting}[language=SQL]
GRANT CREATE SESSION TO user_10g IDENTIFIED BY manager;
GRANT CREATE SESSION TO user_11g IDENTIFIED BY manager;
GRANT CREATE SESSION TO user_12c IDENTIFIED BY manager;
GRANT CREATE SESSION TO user_all IDENTIFIED BY manager;
\end{lstlisting}

Reset all passwords using \emph{IDENTIFIED BY VALUES} to explicitly set
a particular password verifier.

\begin{lstlisting}[language=SQL]
ALTER USER user_10g IDENTIFIED BY VALUES '808E79166793CFD1';
ALTER USER user_11g IDENTIFIED BY VALUES 'S:22D8239017006EBDE054108BF367F
                                        225B5E731D12C91A3BEB31FA28D4A38';
ALTER USER user_12c IDENTIFIED BY VALUES 'T:C6CE7A88CC5D0E048F32A564D2B6A7
                                        BDC78A2092184F28D13A90FC071F804E5E
                                        A09D4D2A3749AA79BFD0A90D18DEC5788D
                                        2B8754AE20EE5C309DBA87550E8AA15EAF
                                        2746ED431BF4543D2ABE33E22678';
\end{lstlisting}

See what we do have in \emph{dba\_users}.

\begin{lstlisting}[language=SQL]
set linesize 160 pagesize 200
col username for a25
SELECT username,password_versions FROM dba_users WHERE username LIKE 'USER_%' ORDER BY 1;

USERNAME          PASSWORD_VERSIONS
------------------------- -----------------
USER_10G          10G
USER_11G          11G
USER_12C          12C
USER_ALL          10G 11G 12C
\end{lstlisting}

See what we do have in \emph{user\$}.

\begin{lstlisting}[language=SQL]
set linesize 160 pagesize 200
col name for a20
col password for a20
col spare4 for a65
SELECT name,password,spare4 FROM user$ 
    WHERE name LIKE 'USER_%' ORDER BY 1;

NAME       PASSWORD          SPARE4
---------- ----------------- --------------------------------------------
USER_10G   808E79166793CFD1
USER_11G                     S:22D8239017006EBDE054108BF367F225B5E731D12C
                             91A3BEB31FA28D4A38
USER_12C                     T:C6CE7A88CC5D0E048F32A564D2B6A7BDC78A209218
                             4F28D13A90FC071F804E5EA09D4D2A3749AA79BFD0A9
                             0D18DEC5788D2B8754AE20EE5C309DBA87550E8AA15E
                             AF2746ED431BF4543D2ABE33E22678

USER_ALL   BFD595809B6149CB  S:804A87EA761505458FDED9B057A77FCF53DA3DDBD6
                             EDB168501EDF5C0B10;T:7950DF0D54DEA24F1764EBC
                             34A262D784E18F4292510B8A2E0D0F7ADFEC1C6F1E22
                             D841A9D91BAF0B9B05632F6D4898C6F4AE1EEF150933
                             9EBCE261A1F36E834A5E2DD9F1E772AB2D6413CCAB5E
                             B0B23
\end{lstlisting}

Check what we do have in \emph{sqlnet.ora}.

\begin{lstlisting}[language=SQL]
host grep -i ALLOWED /u00/app/oracle/network/admin/sqlnet.ora
#SQLNET.ALLOWED_LOGON_VERSION_CLIENT=12a
SQLNET.ALLOWED_LOGON_VERSION_SERVER=11

host sed -i "s|^SQLNET.ALLOWED_LOGON_VERSION_SERVER.*|SQLNET.ALLOWED_LOGON_VERSION_SERVER=11|" \
    /u00/app/oracle/network/admin/sqlnet.ora
host sed -i "s|^SQLNET.ALLOWED_LOGON_VERSION_SERVER.*|SQLNET.ALLOWED_LOGON_VERSION_SERVER=12|" \
    /u00/app/oracle/network/admin/sqlnet.ora
host sed -i "s|^SQLNET.ALLOWED_LOGON_VERSION_SERVER.*|SQLNET.ALLOWED_LOGON_VERSION_SERVER=12a|" \
    /u00/app/oracle/network/admin/sqlnet.ora
\end{lstlisting}

Do some login tests

\begin{lstlisting}[language=SQL]
SQL> connect user_10g/manager
ERROR:
ORA-01017: invalid username/password; logon denied


Warning: You are no longer connected to ORACLE.

connect user_11g/manager
\end{lstlisting}

\hypertarget{setup-kerberos}{%
\subsection{Setup Kerberos}\label{setup-kerberos}}

Check the configuration scripts in \emph{sqlnet.ora}.

\begin{lstlisting}[language=bash]
grep -i -A 11 -B 2 "Kerberos Configuration" $TNS_ADMIN/sqlnet.ora

##########################################################################
# Kerberos Configuration
##########################################################################
SQLNET.AUTHENTICATION_SERVICES = (BEQ,KERBEROS5)
#SQLNET.AUTHENTICATION_SERVICES = (ALL)
SQLNET.FALLBACK_AUTHENTICATION = TRUE
SQLNET.KERBEROS5_KEYTAB = /u00/app/oracle/network/admin/urania.keytab
SQLNET.KERBEROS5_REALMS = /u00/app/oracle/network/admin/krb.realms
SQLNET.KERBEROS5_CC_NAME = /u00/app/oracle/network/admin/krbcache
SQLNET.KERBEROS5_CONF = /u00/app/oracle/network/admin/krb5.conf
SQLNET.KERBEROS5_CONF_MIT=TRUE
SQLNET.AUTHENTICATION_KERBEROS5_SERVICE = oracle
\end{lstlisting}

Check the configuration scripts in \emph{krb5.conf}.

\begin{lstlisting}[language=bash]
cat $TNS_ADMIN/krb5.conf

####krb5.conf DB Server
[logging]
default = FILE:/u00/app/oracle/network/log/krb5lib.log
kdc=FILE:/u00/app/oracle/network/log/krb5kdc.log
admin_server=FILE:/u00/app/oracle/network/log/kadmind.log

[libdefaults]
 default_realm = POSTGASSE.ORG
 clockskew=300
 ticket_lifetime = 24h
 renew_lifetime = 7d
 forwardable = true

[realms]
 POSTGASSE.ORG = {
   kdc = mneme.postgasse.org
   admin_server = mneme.postgasse.org
}

[domain_realm]
.postgasse.org = POSTGASSE.ORG
postgasse.org = POSTGASSE.ORG
\end{lstlisting}

lookup hostname's and check DNS configuration

\begin{lstlisting}[language=bash]
cat /etc/resolv.conf
# Generated by NetworkManager
search aux.lan postgasse.org
nameserver 192.168.56.70
nameserver 10.154.0.1
\end{lstlisting}

\begin{lstlisting}[language=bash]
nslookup mneme.postgasse.org
Server:     192.168.56.70
Address:    192.168.56.70#53

Name:   mneme.postgasse.org
Address: 192.168.56.70
Name:   mneme.postgasse.org
Address: 10.0.2.19
\end{lstlisting}

\begin{lstlisting}[language=bash]
nslookup te2018_eusdb.postgasse.org
Server:     192.168.56.70
Address:    192.168.56.70#53

Name:   urania.postgasse.org
Address: 192.168.56.90
\end{lstlisting}

Create a service principle in MS AD

Create the keytab file

\begin{lstlisting}
ktpass.exe -princ oracle/te2018_eusdb.postgasse.org@POSTGASSE.ORG \
    -mapuser te2018_eusdb.postgasse.org -pass manager \
    -crypto ALL -ptype KRB5_NT_PRINCIPAL \
    -out C:\u00\app\oracle\network\te2018_eusdb.keytab
\end{lstlisting}

Connect as kerberos User \#\# Setup OUD AD Proxy

\hypertarget{requirements}{%
\subsubsection{Requirements}\label{requirements}}

Before you can start you may need a few things.

\begin{itemize}
\tightlist
\item
  Docker environment (eg. Docker community edition)
\item
  OUD Docker Images in particular one for OUD 12.2.1.3 with the latest
  OUD base see \href{https://github.com/oehrlis/docker}{oehrlis/docker}
  soon you may also get the Dockerfiles from the Oracle Repository see
  \href{https://github.com/oracle/docker-images/pull/911}{pull request
  911}
\item
  An MS AD Directory server or at lease a few credential to access one
\end{itemize}

\hypertarget{environment-variable}{%
\subsubsection{Environment Variable}\label{environment-variable}}

To type less you just have to define a few environment variables.
Basically you will define the local Docker volume path, container name,
container hostname and the OUD instance name.

\begin{lstlisting}[language=bash]
export MY_CONTAINER="te2018_oud"
export MY_VOLUME_PATH="/data/docker/volumes/$MY_CONTAINER"
export MY_HOST="$MY_CONTAINER.postgasse.org"
export MY_OUD_INSTANCE="oud_adproxy"
\end{lstlisting}

\hypertarget{create-the-container}{%
\subsubsection{Create the container}\label{create-the-container}}

Just create a container without starting it. Adjust ports, base DN etc.

\begin{lstlisting}[language=bash]
docker container create --name $MY_CONTAINER \
    --volume $MY_VOLUME_PATH:/u01 \
    -p 1389:1389 -p 1636:1636 -p 4444:4444 \
    -e OUD_CUSTOM=TRUE \
    -e BASEDN="dc=postgasse,dc=org" \
    -e OUD_INSTANCE=$MY_OUD_INSTANCE \
    --hostname $MY_HOST \
    --dns 192.168.56.70 \
    --dns-search postgasse.org \
    oracle/oud:12.2.1.3.180626
\end{lstlisting}

Get and configure your create scripts out of the container from the OUD
base. Alternatively you may also get it directly from GitHub
\href{https://github.com/oehrlis/oudbase}{oehrlis/oudbase}.

Get the OUD EUS AD templates from the Docker container created before.

\begin{lstlisting}[language=bash]
mkdir -p $MY_VOLUME_PATH/admin/$MY_OUD_INSTANCE
docker cp \
    $(docker ps -aqf "name=$MY_CONTAINER"):/u00/app/oracle/local/oudbase/templates/create/oud12c_eus_ad_proxy \
    $MY_VOLUME_PATH/admin/$MY_OUD_INSTANCE
mv $MY_VOLUME_PATH/admin/$MY_OUD_INSTANCE/oud12c_eus_ad_proxy $MY_VOLUME_PATH/admin/$MY_OUD_INSTANCE/create
mkdir -p $MY_VOLUME_PATH/admin/$MY_OUD_INSTANCE/etc
echo "manager" >$MY_VOLUME_PATH/admin/$MY_OUD_INSTANCE/etc/${MY_OUD_INSTANCE}_pwd.txt
\end{lstlisting}

Update the \emph{00\_init\_environment} according to your environment.
In particular the variables AD\_PDC\_HOST,AD\_PDC\_PORT, AD\_PDC\_USER,
AD\_PDC\_PASSWORD and BASEDN, GROUP\_DN, USER\_DN

\begin{lstlisting}[language=bash]
vi $MY_VOLUME_PATH/admin/$MY_OUD_INSTANCE/create/00_init_environment

sed -i -e "s|<PDC_HOSTNAME>|mneme.postgasse.org|g" \
    $MY_VOLUME_PATH/admin/$MY_OUD_INSTANCE/create/00_init_environment
sed -i -e 's|<USER_DN>|CN=OUD\\ Admin,CN=Users,dc=postgasse,dc=org|g' \
    $MY_VOLUME_PATH/admin/$MY_OUD_INSTANCE/create/00_init_environment
sed -i -e "s|<PASSWORD>|manager|g" \
    $MY_VOLUME_PATH/admin/$MY_OUD_INSTANCE/create/00_init_environment

sed -i -e 's|^export BASEDN.*|export BASEDN="dc=postgasse,dc=org"|g' \
    $MY_VOLUME_PATH/admin/$MY_OUD_INSTANCE/create/00_init_environment
sed -i -e 's|^export GROUP_OU.*|export GROUP_OU="ou=Groups,dc=postgasse,dc=org"|g' \
    $MY_VOLUME_PATH/admin/$MY_OUD_INSTANCE/create/00_init_environment
sed -i -e 's|^export USER_OU.*|export USER_OU="ou=People,dc=postgasse,dc=org"|g' \
    $MY_VOLUME_PATH/admin/$MY_OUD_INSTANCE/create/00_init_environment
sed -i -e "s|dc=example,dc=com|dc=postgasse,dc=org|g" \
    $MY_VOLUME_PATH/admin/$MY_OUD_INSTANCE/create/00_init_environment

cat $MY_VOLUME_PATH/admin/$MY_OUD_INSTANCE/create/00_init_environment
\end{lstlisting}

Lets go. Start the container and let the scripts create the OUD
instance.

\begin{lstlisting}[language=bash]
docker start $MY_CONTAINER
\end{lstlisting}

Enjoy the log and see how your OUD EUS AD proxy is created

\begin{lstlisting}[language=bash]
docker logs -f $MY_CONTAINER
\end{lstlisting}

\hypertarget{setup-eus}{%
\subsection{Setup EUS}\label{setup-eus}}

\begin{lstlisting}[language=bash]
dbca -configureDatabase -sourceDB $ORACLE_SID -registerWithDirService true \
    -dirServiceUserName "cn=eusadmin" -dirServicePassword manager \
    -walletPassword TVD04manager -silent
\end{lstlisting}

Create a global DB User

\begin{lstlisting}[language=SQL]
DROP USER eus_users;
CREATE USER eus_users IDENTIFIED GLOBALLY;  
GRANT tvd_connect TO eus_users;  
\end{lstlisting}

Define a EUS mapping to the shared schema created before

\begin{lstlisting}[language=bash]
eusm createMapping database_name="$ORACLE_SID" \
    realm_dn="dc=postgasse,dc=org" map_type=SUBTREE \
    map_dn="ou=People,dc=postgasse,dc=org" schema=EUS_USERS \
    ldap_host="te2018_oud.postgasse.org" ldap_port=1389 ldap_user_dn="cn=eusadmin" \
    ldap_user_password="manager"  
\end{lstlisting}

\begin{lstlisting}[language=bash]
eusm listMappings database_name="$ORACLE_SID" \
    realm_dn="dc=postgasse,dc=org" \
    ldap_host="te2018_oud.postgasse.org" ldap_port=1389 ldap_user_dn="cn=eusadmin" \
    ldap_user_password="manager"
\end{lstlisting}

Passwords are in docker logs or in the password files in
\passthrough{\lstinline!$MY\_VOLUME\_PATH/admin/$MY\_OUD\_INSTANCE/etc!}

check EUS connection

\begin{lstlisting}[language=SQL]
SQL> conn dinu/manager
Connected.
SQL> @sousrinf
Database Information
--------------------
- DB_NAME       : TDB122A
- DB_DOMAIN     :
- INSTANCE      : 1
- INSTANCE_NAME     : TDB122A
- SERVER_HOST       : urania
-
Authentification Information
----------------------------
- SESSION_USER      : EUS_USERS
- PROXY_USER        :
- AUTHENTICATION_METHOD : PASSWORD
- IDENTIFICATION_TYPE   : GLOBAL SHARED
- NETWORK_PROTOCOL  :
- OS_USER       : oracle
- AUTHENTICATED_IDENTITY: DINU
- ENTERPRISE_IDENTITY   : cn=Martin Berger,ou=People,dc=postgasse,dc=org
-
Other Information
-----------------
- ISDBA         : FALSE
- CLIENT_INFO       :
- PROGRAM       : sqlplus@urania (TNS V1-V3)
- MODULE        : SQL*Plus
- IP_ADDRESS        :
- SID           : 33
- SERIAL#       : 17568
- SERVER        : DEDICATED
- TERMINAL      : pts/1

PL/SQL procedure successfully completed.
\end{lstlisting}



\end{document}
%% - EOF -----------------------------------------------------------------