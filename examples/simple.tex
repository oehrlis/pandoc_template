\hypertarget{requirements-workshop-environment}{%
\section{Requirements Workshop
Environment}\label{requirements-workshop-environment}}

During the workshop \textbf{\emph{O-DB-DOCKER}} \emph{Oracle-Databases
in Docker-Containers} there will be the possibility to explore different
topics with practical examples. Each participant will be provided with a
compute node in the Oracle Cloud for the duration of the DOAG training
day. Alternatively, participants can perform the exercises in their own
local VM or Docker environment. Participants are free to choose which
environment they want to use for the workshop. Although the setup of the
local VM respectively local Docker environment is not part of the
workshop itself. The following summary gives a short overview of the
different requirements for the three workshop environments.

Detailed information on the workshop environment, documents,
instructions etc. are available prior to the DOAG Training Day via
\href{https://url.oradba.ch/DOAG2019_O-DB-DOCKER}{DOAG2019 O-DB-DOCKER}.
If you have any questions, don't hesitate to stop by the Trivadis stand
and ask for Stefan Oehrli.

\hypertarget{requirements-and-skills}{%
\subsection{Requirements and Skills}\label{requirements-and-skills}}

The different exercises of the workshop allow a step-by-step
introduction to the topic \emph{Oracle databases in Docker containers}.
The following knowledge of the participants is recommended:

\begin{itemize}
\tightlist
\item
  Oracle database basics like installation, configuration and basic
  database administration
\item
  Docker basics (see also
  \href{https://docs.docker.com/get-started/}{Get Started, Part 1:
  Orientation and setup})
\item
  Practical experience with shell scripts, SSH and the command line.
\end{itemize}

\hypertarget{compute-node-in-the-oracle-cloud}{%
\subsection{Compute Node in the Oracle
Cloud}\label{compute-node-in-the-oracle-cloud}}

The Compute Node in the Oracle Cloud will be specially prepared for this
workshop and will be available for practical work for the duration of
the DOAG training day. Each compute node is configured as follows:

\begin{itemize}
\tightlist
\item
  \emph{Host name:} ol7dockerXX.trivadislabs.com (see host overview on
  \href{https://url.oradba.ch/DOAG2019_O-DB-DOCKER}{DOAG2019
  O-DB-DOCKER})
\item
  \emph{Internal IP address:} 10.0.0.2
\item
  \emph{External IP address:} see host overview on
  \href{https://url.oradba.ch/DOAG2019_O-DB-DOCKER}{DOAG2019
  O-DB-DOCKER}
\item
  \emph{VM shape:} VM.Standard2.2

  \begin{itemize}
  \tightlist
  \item
    \emph{CPU:} 2.0 GHz Intel® Xeon® Platinum 8167M (2 Cores)
  \item
    \emph{Memory:} 30GB
  \item
    \emph{Disk:} ca 256GB
  \end{itemize}
\item
  \emph{Software:}

  \begin{itemize}
  \tightlist
  \item
    Oracle Enterprise Linux 7.7
  \item
    Docker Engine / Community Edition
  \item
    Predefined Docker Images
  \item
    Miscellaneous Oracle binaries and Git client
  \end{itemize}
\end{itemize}

Access to the compute nodes is exclusively via SSH and Private Keys.
Workshop participants must ensure that they meet the following
requirements:

\begin{itemize}
\tightlist
\item
  \emph{SSH client} for remote access, e.g.~Putty, MobaXterm or similar.
\item
  \emph{SCP Client} to copy files remotely, e.g.~WinSCP, Putty or
  similar.
\item
  \emph{Text editor} for customizing / developing docker files, scripts
  etc. e.g.~MS Visual Studio Code, UltraEdit, Notepad++ or similar
\item
  It must also be ensured that access to a public IP address or host
  name is possible via an SSH key.
\end{itemize}

The following \emph{optional} points are recommended:

\begin{itemize}
\tightlist
\item
  GitHub account to access and download the source code. Simple download
  does not require an account.
\end{itemize}

\hypertarget{local-vagrant-vm}{%
\subsection{Local Vagrant VM}\label{local-vagrant-vm}}

As with compute nodes, all exercises can be performed directly in a
Local VM. Appropriate vagrant scripts for building a VM are available in
the Git Repository
\href{https://github.com/oehrlis/o-db-docker}{oehrlis/o-db-docker}. The
following requirements must be met in order to set up this VM with
Vagrant:

\begin{itemize}
\tightlist
\item
  \href{https://www.virtualbox.org/wiki/Downloads}{Virtualbox}
\item
  \href{https://www.vagrantup.com}{Vagrant}
\item
  Local clone of the Git repository
  \href{https://github.com/oehrlis/o-db-docker}{oehrlis/o-db-docker}
\item
  Oracle Binaries for Oracle 19c and current RU.
\item
  Sufficient hard disk space for the VM and the Docker Images approx.
  50GB
\item
  If necessary, additional tools to access and work with the VM,
  e.g.~SSH client, text editor, etc.
\end{itemize}

Setting up a local VM is not part of the workshop. Participants who wish
to work with a VM must configure it in advance.

\hypertarget{local-docker-environment}{%
\subsubsection{Local Docker
Environment}\label{local-docker-environment}}

As a third option, the exercises can also be performed in a local docker
environment. This is especially useful for working on Linux or MacOS
notebooks. In order to perform the workshop locally, the following
requirements must be met:

\begin{itemize}
\tightlist
\item
  Installing the Docker Community Edition. See also
  \href{https://docs.docker.com/install/}{About Docker - Community}
\item
  Local clone of the Git repository
  \href{https://github.com/oehrlis/o-db-docker}{oehrlis/o-db-docker} and
  \href{https://github.com/oracle/docker-images}{oracle/docker-images)}
\item
  Oracle Binaries for Oracle 19c and current RU.
\item
  Sufficient hard disk space for the VM and the Docker Images approx.
  50GB
\item
  If necessary, additional tools to access and work with the VM,
  e.g.~SSH client, text editor, git client etc.
\end{itemize}

Building a local Docker environment is not part of the workshop.
Participants who wish to work with a local Docker installation must
configure it in advance.
